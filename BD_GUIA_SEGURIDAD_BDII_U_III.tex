\documentclass[a4paper,twocolumn,10pt]{article}
\usepackage[spanish]{babel}
\usepackage[latin1]{inputenc}
\usepackage{graphicx}
\usepackage{flushend}
\usepackage{enumerate}

\begin{document}

\title{Guia de Seguridad de Base de Datos}
\author{\begin{tabular}{c c c c c }
Brandon M. &  Angelo Q. & Jhordy V. & Leidy H. & Angela B. \\
(2015052715) & (2015052826) & (2015052719) & (2015053230) & (2016054494)\\
\end{tabular}
\\}

\date{}

\twocolumn[
\begin{@twocolumnfalse}
\maketitle
\vspace*{-1cm}
\begin{center}\rule{0.9\textwidth}{0.1mm} \end{center}



\begin{abstract}
\normalsize La seguridad de la base de datos se refiere al uso de una amplia gama de controles de seguridad de la informaci\'on para proteger las bases de datos (que pueden incluir los datos, las aplicaciones de la base de datos o las funciones almacenadas, los sistemas de la base de datos, los servidores de la base de datos y los enlaces de red asociados) contra el compromiso de su confidencialidad, integridad y disponibilidad. Implica varios tipos o categor\'ias de controles, tales como t\'ecnicos, de procedimiento / administrativos y f\'isicos. La seguridad de la base de datos es un tema especializado dentro de los \'ambitos m\'as amplios de la seguridad inform\'atica , la seguridad de la informaci\'on y la gesti\'on de riesgos. 
\begin{center}\rule{0.9\textwidth}{0.1mm} \end{center}
\vspace*{0.5cm}
\end{abstract}
\end{@twocolumnfalse}
]

\section{Introducci\'on}
\section{Objetivos}
\section{Marco Te\'orico}
\section{Desarrollo}
\section{Referencias}
\section{Conclusi\'on}



\end{document}